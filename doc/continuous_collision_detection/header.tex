%\documentclass[11pt,%%draft,
%a4paper,liststotoc,bibtotocnumbered,tablecaptionabove,titlepage,headsepline,2.1headlines, headinclude]{scrreprt} %% mit Kapiteln

%
%  Einbinden der Pakete
% ======================
%
\usepackage{amsmath}                % zusätzliche Mathematische Zeichen
\usepackage{amssymb}                % zusätzliche Mathematische Symbole
\usepackage{float}                  % weitere Möglichkeit zur Plazierung von Floatobjekten mittels "H", das Floatobjekt wird an genau die Stelle gesetzt, die angegeben wird. 
\usepackage{graphicx,color}         % Einbinden von Bildformaten wie jpeg
\usepackage[hyperindex]{hyperref}   % Hyperlinks fuer die elektronische Version
\usepackage{subfigure}              % Erstellen von Unterabbildungen mit eigenen Unterschriften

%  Index (Stichwortverzeichnis)
% ==============================

%\usepackage{index}
%\newindex{default}{idx}{ind}{Namen- und Sachverzeichnis}

%  weitere Pakete, hier nicht verwendet
% ======================================

%\usepackage{a4wide}                % die Seiten nicht im letter Format machen
%\usepackage{amscd}
%\usepackage{array}
%\usepackage[ngerman]{babel}         % damit z.B. Abbildungen dt. sind
%\usepackage[hang]{caption}         % Layoutänderungen von Caption's
%\usepackage{colortab}
%\usepackage{ecltree}               % Zeichnen von Baumdiagrammen
%\usepackage{enumerate}             % Neudefinition der enumerate-Umgebung
%\usepackage[dvips]{epsfig}          % Einbinden von EPS Bildern
%\usepackage{fancyhdr}              % fancyheadings
%\usepackage{floatfig}
%\usepackage[T1]{fontenc}
%\usepackage{hhline}                % Tabellenformatierung
%\usepackage[breaklinks=true]{hyperref} % Hyperrefs in PDF Dokumenten
%\usepackage[hyperindex]{hyperref}  % Hyperlinks fuer die gedruckte Version
%\usepackage{indentfirst}           % 1. Zeile nach Überschrift einrücken
%\usepackage[latin1]{inputenc}       % deutsche Umlaute
%\usepackage{latexsym}              % Einbinden von Latex-Symbolen
%\usepackage{letterspace}
%\usepackage[norules]{lgrind}       % Programmcode Darstellung
%\usepackage{longtable}             % für Tabellen die größer als eine Seite sind
%\usepackage{lscape}                % für einzelne Seiten Landscape
%\usepackage{mathrsfs}
%\usepackage{moreverb}              % verbatim Umgebung nutzen können
%\usepackage{nassi}                 % Nassi-Schneidermann Diagramme
%\usepackage{ngerman}                % neue deutsche Rechtschreibung, z.B. für Silbentrennung
%\usepackage{psfrag}                % Beschriftungen in EPS Bildern austauschen, nur mit dem Package graphicx moeglich
%\usepackage[]{scrpage2}            % fuer die Kopf- und Fusszeilen erforderlich (KOMA-Script)
%\usepackage{setspace}              % fuer größere Zeilenabstände
%\usepackage{supertabular}          % Tabelle über mehrere Seiten zulassen
%\usepackage{tabularx}              % automatisches Berechnen der Spaltenbreite
%\usepackage{theorem}
%\usepackage{thumbpdf}              % pdf-Thumbnails
%\usepackage{times}                 % Schriftart Times
%\usepackage{wrapfig}
%\usepackage[arc,web,arrow]{xy}     % xy-pic Umgebung für einfache Bilder

%  Einzug und Zeilenabstand verändern
% ====================================

%\typearea[1.5cm]{11}                %% gibt den beschreibbaren Textbereich an. (KOMA-Script)
%\setlength{\parindent}{1em}         %% Einzug bei neuem Absatz auf 0 setzen
%\setlength{\parskip}{0mm}          %% Abstand zwischen Absätzen auf 0 setzen
%   1.5-fachen Zeilenabstand
%\onehalfspacing
%   2-fachen Zeilenabstand
%\doublespacing

%  Ränder definieren
% ===================

%\headheight1cm
%\oddsidemargin0cm                      %% linker Rand
%\evensidemargin0cm                     %% linker Rand
%\textwidth15.5cm                       %% Textbreite
%\textheight24.4cm                      %% Gesamthöhe für den Seitentext
%\topmargin-1cm                         %% Oberer Rand bis Oberkante Kopfzeile

%\setlength{\oddsidemargin}{-5mm}
%\setlength{\voffset}{-25mm}
%\setlength{\hoffset}{0mm}
%\setlength{\topmargin}{-15mm}
%\setlength{\linewidth}{160mm}
%\setlength{\textheight}{225mm}
%\setlength{\headsep}{10mm}
%\setlength{\footskip}{15mm}

%  Pfade und Dateinamenerweiterung für Bilder
% ============================================

% \graphicspath{{./Bilder/}{../../Projekt/Ergebnisse/}}
%\DeclareGraphicsExtensions{.eps,.jpg}


%  Alternatives Headerformat, für Reports sinnvoll
% =================================================

%\pagestyle{fancy}
%\pagestyle{headings}
%
%\renewcommand{\chaptermark}[1]{\markboth{\thechapter\ #1}{}}
%\renewcommand{\sectionmark}[1]{\markright{#1}}
%\renewcommand{\chapterpagestyle}{fancy}         %% definiert den Seitenstil bei Kapitelanfängen neu
%\lhead[]{\leftmark}
%\chead{}
%\rhead[]{\thepage}
%\lfoot[]{}
%\cfoot[]{}
%\rfoot[]{}
%\renewcommand{\headrulewidth}{0.2pt}
%\renewcommand{\footrulewidth}{0.2pt}

%  Schrift im mathematischen Modus in verschiedenen Größen
% =========================================================

\newcommand{\mathtiny}[1]{\text{\tiny\ensuremath{#1}}}
\newcommand{\mathscriptsize}[1]{\text{\scriptsize\ensuremath{#1}}}
\newcommand{\mathfootnotesize}[1]{\text{\footnotesize\ensuremath{#1}}}
\newcommand{\mathsmall}[1]{\text{\small\ensuremath{#1}}}
\newcommand{\mathnormalsize}[1]{\text{\normalsize\ensuremath{#1}}}
\newcommand{\mathlarge}[1]{\text{\large\ensuremath{#1}}}
\newcommand{\mathLarge}[1]{\text{\Large\ensuremath{#1}}}
\newcommand{\mathLARGE}[1]{\text{\LARGE\ensuremath{#1}}}
\newcommand{\mathhuge}[1]{\text{\huge\ensuremath{#1}}}
\newcommand{\mathHuge}[1]{\text{\Huge\ensuremath{#1}}}
\newcommand{\mathframe}[1]{\text{\frame{\ensuremath{#1}}}}

%  Einige trigonometrische Funktionen als Operatoren
% ===================================================

\newcommand{\arcsinh}{\operatorname{arcsinh}}
\newcommand{\arccosh}{\operatorname{arccosh}}
\newcommand{\arctanh}{\operatorname{arctanh}}
\newcommand{\arccot}{\operatorname{arccot}}
\newcommand{\arccoth}{\operatorname{arccoth}}

