\documentclass[a4paper,10pt]{scrartcl}

%\documentclass[11pt,%%draft,
%a4paper,liststotoc,bibtotocnumbered,tablecaptionabove,titlepage,headsepline,2.1headlines, headinclude]{scrreprt} %% mit Kapiteln

%
%  Einbinden der Pakete
% ======================
%
\usepackage{amsmath}                % zusätzliche Mathematische Zeichen
\usepackage{amssymb}                % zusätzliche Mathematische Symbole
\usepackage{float}                  % weitere Möglichkeit zur Plazierung von Floatobjekten mittels "H", das Floatobjekt wird an genau die Stelle gesetzt, die angegeben wird. 
\usepackage{graphicx,color}         % Einbinden von Bildformaten wie jpeg
\usepackage[hyperindex]{hyperref}   % Hyperlinks fuer die elektronische Version
\usepackage{subfigure}              % Erstellen von Unterabbildungen mit eigenen Unterschriften

%  Index (Stichwortverzeichnis)
% ==============================

%\usepackage{index}
%\newindex{default}{idx}{ind}{Namen- und Sachverzeichnis}

%  weitere Pakete, hier nicht verwendet
% ======================================

%\usepackage{a4wide}                % die Seiten nicht im letter Format machen
%\usepackage{amscd}
%\usepackage{array}
%\usepackage[ngerman]{babel}         % damit z.B. Abbildungen dt. sind
%\usepackage[hang]{caption}         % Layoutänderungen von Caption's
%\usepackage{colortab}
%\usepackage{ecltree}               % Zeichnen von Baumdiagrammen
%\usepackage{enumerate}             % Neudefinition der enumerate-Umgebung
%\usepackage[dvips]{epsfig}          % Einbinden von EPS Bildern
%\usepackage{fancyhdr}              % fancyheadings
%\usepackage{floatfig}
%\usepackage[T1]{fontenc}
%\usepackage{hhline}                % Tabellenformatierung
%\usepackage[breaklinks=true]{hyperref} % Hyperrefs in PDF Dokumenten
%\usepackage[hyperindex]{hyperref}  % Hyperlinks fuer die gedruckte Version
%\usepackage{indentfirst}           % 1. Zeile nach Überschrift einrücken
%\usepackage[latin1]{inputenc}       % deutsche Umlaute
%\usepackage{latexsym}              % Einbinden von Latex-Symbolen
%\usepackage{letterspace}
%\usepackage[norules]{lgrind}       % Programmcode Darstellung
%\usepackage{longtable}             % für Tabellen die größer als eine Seite sind
%\usepackage{lscape}                % für einzelne Seiten Landscape
%\usepackage{mathrsfs}
%\usepackage{moreverb}              % verbatim Umgebung nutzen können
%\usepackage{nassi}                 % Nassi-Schneidermann Diagramme
%\usepackage{ngerman}                % neue deutsche Rechtschreibung, z.B. für Silbentrennung
%\usepackage{psfrag}                % Beschriftungen in EPS Bildern austauschen, nur mit dem Package graphicx moeglich
%\usepackage[]{scrpage2}            % fuer die Kopf- und Fusszeilen erforderlich (KOMA-Script)
%\usepackage{setspace}              % fuer größere Zeilenabstände
%\usepackage{supertabular}          % Tabelle über mehrere Seiten zulassen
%\usepackage{tabularx}              % automatisches Berechnen der Spaltenbreite
%\usepackage{theorem}
%\usepackage{thumbpdf}              % pdf-Thumbnails
%\usepackage{times}                 % Schriftart Times
%\usepackage{wrapfig}
%\usepackage[arc,web,arrow]{xy}     % xy-pic Umgebung für einfache Bilder

%  Einzug und Zeilenabstand verändern
% ====================================

%\typearea[1.5cm]{11}                %% gibt den beschreibbaren Textbereich an. (KOMA-Script)
%\setlength{\parindent}{1em}         %% Einzug bei neuem Absatz auf 0 setzen
%\setlength{\parskip}{0mm}          %% Abstand zwischen Absätzen auf 0 setzen
%   1.5-fachen Zeilenabstand
%\onehalfspacing
%   2-fachen Zeilenabstand
%\doublespacing

%  Ränder definieren
% ===================

%\headheight1cm
%\oddsidemargin0cm                      %% linker Rand
%\evensidemargin0cm                     %% linker Rand
%\textwidth15.5cm                       %% Textbreite
%\textheight24.4cm                      %% Gesamthöhe für den Seitentext
%\topmargin-1cm                         %% Oberer Rand bis Oberkante Kopfzeile

%\setlength{\oddsidemargin}{-5mm}
%\setlength{\voffset}{-25mm}
%\setlength{\hoffset}{0mm}
%\setlength{\topmargin}{-15mm}
%\setlength{\linewidth}{160mm}
%\setlength{\textheight}{225mm}
%\setlength{\headsep}{10mm}
%\setlength{\footskip}{15mm}

%  Pfade und Dateinamenerweiterung für Bilder
% ============================================

% \graphicspath{{./Bilder/}{../../Projekt/Ergebnisse/}}
%\DeclareGraphicsExtensions{.eps,.jpg}


%  Alternatives Headerformat, für Reports sinnvoll
% =================================================

%\pagestyle{fancy}
%\pagestyle{headings}
%
%\renewcommand{\chaptermark}[1]{\markboth{\thechapter\ #1}{}}
%\renewcommand{\sectionmark}[1]{\markright{#1}}
%\renewcommand{\chapterpagestyle}{fancy}         %% definiert den Seitenstil bei Kapitelanfängen neu
%\lhead[]{\leftmark}
%\chead{}
%\rhead[]{\thepage}
%\lfoot[]{}
%\cfoot[]{}
%\rfoot[]{}
%\renewcommand{\headrulewidth}{0.2pt}
%\renewcommand{\footrulewidth}{0.2pt}

%  Schrift im mathematischen Modus in verschiedenen Größen
% =========================================================

\newcommand{\mathtiny}[1]{\text{\tiny\ensuremath{#1}}}
\newcommand{\mathscriptsize}[1]{\text{\scriptsize\ensuremath{#1}}}
\newcommand{\mathfootnotesize}[1]{\text{\footnotesize\ensuremath{#1}}}
\newcommand{\mathsmall}[1]{\text{\small\ensuremath{#1}}}
\newcommand{\mathnormalsize}[1]{\text{\normalsize\ensuremath{#1}}}
\newcommand{\mathlarge}[1]{\text{\large\ensuremath{#1}}}
\newcommand{\mathLarge}[1]{\text{\Large\ensuremath{#1}}}
\newcommand{\mathLARGE}[1]{\text{\LARGE\ensuremath{#1}}}
\newcommand{\mathhuge}[1]{\text{\huge\ensuremath{#1}}}
\newcommand{\mathHuge}[1]{\text{\Huge\ensuremath{#1}}}
\newcommand{\mathframe}[1]{\text{\frame{\ensuremath{#1}}}}

%  Einige trigonometrische Funktionen als Operatoren
% ===================================================

\newcommand{\arcsinh}{\operatorname{arcsinh}}
\newcommand{\arccosh}{\operatorname{arccosh}}
\newcommand{\arctanh}{\operatorname{arctanh}}
\newcommand{\arccot}{\operatorname{arccot}}
\newcommand{\arccoth}{\operatorname{arccoth}}


\newcommand{\abs}[1]{\vert{#1}\vert}
\newcommand{\norm}[1]{\left\Vert {#1} \right\Vert}

\setcounter{tocdepth}{2}
\setcounter{secnumdepth}{2}

\begin{document}

% \tableofcontents
% \listoffigures
% \addcontentsline{toc}{chapter}{\listfigurename}

\section{Circle--Circle Collision}

% \begin{figure}[htb]
%     \centering
%     \input{circ_circ_01.pdf_t}
% \end{figure}
\noindent
\textbf{Definitions:}\\

\noindent
Position circle A: $\vec{p}_A$ \\
Position circle B: $\vec{p}_B$ \\
$\vec{p}_{A0} = \vec{p}_A(t=0)$ \\
$\vec{p}_{A1} = \vec{p}_A(t=1)$ \\
$\vec{p}_{B0} = \vec{p}_B(t=0)$ \\
$\vec{p}_{B1} = \vec{p}_B(t=1)$ \\
Radius circle 1: $r_1$ \\
Radius circle 2: $r_2$ \\

\noindent
For Circle--Circle collsion detection two cases are need to be
evaluated:
\begin{enumerate}
    \item Circles do not cover a common area
    \item One circle is inside the other
\end{enumerate}
For the calculation, this only changes the way the radii are
handled. For case 1, the following definition is used
\begin{equation}
    r = r_1+r_2
\end{equation}
Case 2 implies
\begin{equation}
    r = \abs{r_1-r_2}
\end{equation}
With~$t$ being the time of collision factor ($t \in [0,1]]$) the following equation must be solved:
\begin{equation}
    \norm{\vec{p}_{A0} + ( \vec{p}_{A1} - \vec{p}_{A0} ) \cdot t -
    (
    \vec{p}_{B0} + ( \vec{p}_{B1} - \vec{p}_{B0} ) \cdot t
    )
    }
    = r
\end{equation}
\begin{equation}
    \norm{\vec{p}_{A0} + \vec{p}_{\Delta A} \cdot t -
    (
    \vec{p}_{B0} + \vec{p}_{\Delta B} \cdot t
    )
    }
    = r
\end{equation}
\begin{equation}
    \norm{(\vec{p}_{A0} - \vec{p}_{B0}) +
    (
     \vec{p}_{\Delta A} - \vec{p}_{\Delta B}
    )
    \cdot t
    }
    = r
\end{equation}
\begin{equation}
    [(\vec{p}_{A0} - \vec{p}_{B0}) +
    (
     \vec{p}_{\Delta A} - \vec{p}_{\Delta B}
    )
    \cdot t
    ]^2
    = r^2
\end{equation}
\begin{equation}
    [U + V \cdot t]^2
    = r^2
\end{equation}
\begin{equation}
    U^2 + 2 UVt + V^2 t^2
    = r^2
\end{equation}
\begin{equation}
    t^2 + 2\frac{UV}{V^2} t + \frac{U^2-r^2}{V^2} = 0
\end{equation}
\begin{equation}
    t_{1/2} = -\frac{p}{2} \qquad\stackrel{+}{-} \qquad\sqrt{\left(\frac{p}{2}\right)^2 - q}
\end{equation}
with
\begin{equation}
    \frac{p}{2} = \frac{UV}{V^2}
\end{equation}
and
\begin{equation}
    q = \frac{U^2-r^2}{V^2}
\end{equation}
If the square root is complex or~$t$ is negative, there is no collsion. If the square root is zero,
there is one collision.
Otherwise, there are two solutions for~$t$, first collision appears for the smaller~$t$.

\section{Polygon--Polygon Collision}

For polygon--polygon collision every vertex~$v_n$ of polygon~$B$ is tested for collision
with every line between vertex~$v_n$ and~$v_{n+1}$ of polygon~$A$.\\

\noindent
\textbf{Definitions:}\\

\noindent
Point on line between vertex~$v_n$ and~$v_{n+1}$ of polygon~$A$ for each dimension: $p_A$ \\
Vertex~$v_n$ of polygon~$B$ for each dimension: $p_B$ \\
$p_{A0} = p_A(t=0)$ \\
$p_{A1} = p_A(t=1)$ \\
$p_{B0} = p_B(t=0)$ \\
$p_{B1} = p_B(t=1)$ \\

\subsection{Version 1 -- Point--Line}

The line between vertices of polygon~$A$ is defined by the accordant vertices~$p_{A00} = 
p_{A_0}(t=0)$ and~$p_{A10} = p_{A_1}(t=0)$
\begin{equation}
    p_{A0} = p_{A00} + \alpha (p_{A10} - p_{A00} )
\end{equation}
\begin{equation}
    p_{A1} = p_{A01} + \alpha (p_{A11} - p_{A01} )
\end{equation}
\begin{equation}
    p_{A0} + ( p_{A1} - p_{A0} ) \cdot t = 
    p_{B0} + ( p_{B1} - p_{B0} ) \cdot t
\end{equation}
\begin{equation}
    \frac{p_{A0} - p_{B0}}
    {(p_{B1} - p_{B0}) -
     (p_{A1} - p_{A0})}
     = t
\end{equation}
\begin{equation}
    \frac{p_{A00} + \alpha (p_{A10} - p_{A00} ) -
          p_{B0}}
    {(p_{B1} - p_{B0}) -
     [(p_{A01} + \alpha (p_{A11} - p_{A01} )) -
      (p_{A00} + \alpha (p_{A10} - p_{A00} ))]
    }
     = t
\end{equation}
\begin{equation}
    \frac{p_{A00} - p_{B0} + 
          \alpha (p_{A10} - p_{A00} )
         }
         {(p_{B1} - p_{B0}) -
            (p_{A01} - p_{A00}) +
            \alpha [(p_{A10} - p_{A00} ) - 
                    (p_{A11} - p_{A01} )
                   ]
         }
     = t
\end{equation}
Substitution of constants:
\begin{equation}
    f(\alpha) = 
    \frac{a + \alpha b}
         {c + \alpha d}
     = t
\end{equation}
This is true for coordinates~$x,y$. For both dimensions, time~$t$ must be equal. Hence, $\alpha$
can be calculated by solving the following equation:
\begin{equation}
    \frac{a_x + \alpha b_x}
         {c_x + \alpha d_x}
     =
    \frac{a_y + \alpha b_y}
         {c_y + \alpha d_y}
\end{equation}
After some rearranging we get
\begin{equation}
    \alpha^2 ( b_x d_y - b_y d_x) + 
    \alpha   ( a_x d_y + b_x c_y - a_y d_x - b_y c_x) +
             ( a_x c_y - a_y c_x) = 0
\end{equation}
which is a simple quadratic equation. All valid~$\alpha \in [0,1]$ result in a
time~$t$, that is also only valid for $t \in [0,1]$.

\subsection{Version 1 -- Point--Line/Circle--Line}

\subsubsection{Testing the end points}
The first part is to test the end points of the line for collsion when dealing with a
circle.\\
\begin{equation}
    \norm{
      (\vec{p}_{A0} + t\cdot (\vec{p}_{A1} - \vec{p}_{A0})) -
      (\vec{p}_{B0} + t\cdot (\vec{p}_{B1} - \vec{p}_{B0}))
    }
    = r
\end{equation}
\begin{equation}
    \norm{
      (\vec{p}_{A0}-\vec{p}_{B0} + t\cdot
      (\vec{p}_{A1} - \vec{p}_{A0} - \vec{p}_{B1} + \vec{p}_{B0})
    }
    = r
\end{equation}
\begin{equation}
    \norm{ a + t\cdot b } = r
\end{equation}
\begin{equation}
    b^2 t^2 + 2abt + a^2-r^2 = 0
\end{equation}
This is a simple quadratic equation which must be solved for~$t$.

\subsubsection{Testing the lines between end points}
In this second part, collision is handled by the distance~$d_i$ of a point to the line between
two points. For point--line collision distance~$d_i=0$ is sufficient.
In case of a circle, $d_i=r$ with $r$ being the radius of the circle.\\
The distance to the line is calculated similar to using a plane in Hessian Normal Form in 3D.\\
One approximation is very important: Since a simple interpolation is used, lines do not neccessarily
keep their length. Hence, the norm of a vector between to points might change for this model,
while in reality, it would stay the same. For easier calculation, the unchanged length
is used for calculation, which slightly disagrees with the model.
\begin{equation}
    \norm{\vec{p}_{A10} - \vec{p}_{A00}}
    \approx
    \norm{(\vec{p}_{A10} + t\cdot (\vec{p}_{A11} - \vec{p}_{A10})) -
          (\vec{p}_{A00} + t\cdot (\vec{p}_{A01} - \vec{p}_{A00}))}
\end{equation}
With this, the distance is calculated as follows:
\begin{eqnarray}
    \frac{(\vec{p}_{A10} + t\cdot (\vec{p}_{A11} - \vec{p}_{A10})) -
          (\vec{p}_{A00} + t\cdot (\vec{p}_{A01} - \vec{p}_{A00}))}
         {\norm{\vec{p}_{A10} - \vec{p}_{A00}}} & \times & \\
    \frac{(\vec{p}_{B0} + t \cdot (\vec{p}_{B1} - \vec{p}_{B0})) -
          (\vec{p}_{A00} + t\cdot (\vec{p}_{A01} - \vec{p}_{A00}))}
         {\norm{\vec{p}_{A10} - \vec{p}_{A00}}} & = & d_i
\end{eqnarray}
\begin{eqnarray}
    [(\vec{p}_{A10}-\vec{p}_{A00}) + 
     t\cdot (\vec{p}_{A11} - \vec{p}_{A10} - \vec{p}_{A01} + \vec{p}_{A00})]
          & \times & \\
    \left[(\vec{p}_{B0} - \vec{p}_{A00}) +
          t\cdot (\vec{p}_{B1} - \vec{p}_{B0} - \vec{p}_{A01} + \vec{p}_{A00})\right]
          & = & d_i \norm{\vec{p}_{A10} - \vec{p}_{A00}}
\end{eqnarray}
\begin{equation}
    (a + t\cdot b) \times (c + t\cdot d) = d_i e
\end{equation}
\begin{equation}
    (b \times d)t^2 + (b \times c + a \times d)t + (a \times c - d_i e) = 0
\end{equation}
This is a simple quadratic equation which must be solved for~$t$. There is a
special case: If the line does not move in time, then $b=0$. Hence, the resulting
equation is simplified to:
\begin{equation}
    (a \times d)t + (a \times c - d_i e) = 0
\end{equation}
\begin{equation}
    t = \frac{d_i e -  a \times c}{a \times d}
\end{equation}
In both cases, the order of operands of the cross product must also be changed,
so there are several solutions for time~$t$ to test for the smallest value.
If a time is found, it has to be validated. By using the time and the dot product,
it is tested, if the point of collsion lies within the line.\\

\end{document}